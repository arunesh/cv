\documentclass[11pt,letterpaper]{article}
\usepackage{graphicx}
\usepackage{times}

\newfont{\ttlfnt}{phvb at 18pt}
\newfont{\smttlit}{phvbo at 14pt}
\textwidth=6.99in
\textheight=9.25in \columnwidth=3.33in
\columnsep 0.33in          %    Space between columns
\hoffset=-0.25in \voffset=-0.125in \oddsidemargin=0in \topmargin=0in \headheight=0in \headsep=0in

\def\bull{\vrule height 0.8ex width .7ex depth -.1ex }
\def\mybull{$\bullet$\hspace{2mm}}
%


\usepackage{url}
\usepackage{amsfonts}
\usepackage{subfigure}
\usepackage{amssymb}
\usepackage{amsxtra}
\usepackage{theorem}
\usepackage{algorithm}
\usepackage{algorithmic}
\usepackage{times}
\usepackage{amsmath}


\title{
\vspace{-2cm}Research Statement for Arunesh Mishra\\
{\large Postdoc Research Scientist, University of Wisconsin, Madison.}}

%\title{{\ttlfnt Research Statement for Arunesh Mishra}\\
%{\smttlit Postdoc Research Scientist, University of Wisconsin, Madison.}}

\begin{document}
\date{}
\maketitle
\vspace{-1cm}
We have a choice of a multitude of wireless technologies today, each optimized for a single and distinct service model -- for example, cellular for
voice and WiFi for data. The next challenge is thus, convergence, which places the user in control  by 
allowing for a user-driven model of service rather than the technology dictating what services are available. My thesis has examined one part of this puzzle --
%Predominantly,  people today use
%the cellular technology for voice and 802.11 (or WiFi) for data communication. The next step is to provide a unified
%model to the user for both voice and data - thats is, convergence. Voice-over-IP provides the software-level protocols
%for implementing voice communication over data networks. My thesis addresses one critical piece of this puzzle --
understanding the challenges faced by voice applications when using WiFi. The focus has been on handoffs and the related
latencies faced by users that are mobile within an 802.11 network. A unique aspect of my thesis work has been the design of
practical system-level solutions which work transparently with the standards in reducing the handoff latencies from 1.2
seconds to under 50 ms. These solutions are based on the notion of neighbor graphs - a
structure that captures the topology of a wireless network from the perspective of user mobility. We believe that this
concept could have interesting applications beyond handoffs, for example, in providing better quality-of-service, aiding
network management and improving network security. 

 A next broad direction of my work has been in optimizing the performance of WLANs by building practical and
intelligent channel assignment algorithms. One unique contribution of this work is the concept of client-driven network
management for WLANs and construction of protocol-level methods to realize this. We show that client participation in
channel assignment and for network management in general, is critical for operating the network efficiently at peak
performance. We have also built distributed algorithms to manage 802.11 channels in an uncoordinated wireless
environment, for example, 802.11 access points (APs) belonging to individual hotspots and apartment homes  which interfere and co-exist with
each other. These distributed algorithms perform a fair and near-optimal division of the system's resources using the concept of
channel hopping.  Finally, through careful observations and analysis we've also debunked the traditional approach of
avoiding partially overlapped channels which have been defined by the IEEE standards body.  We've shown that by careful modeling of the interference between two partially
overlapped channels, one can utilize these channels effectively, thus improving the overall system throughput by a
factor of three.

With this experience, I hope to employ my skills and expertise to new and challenging research problems in the broad
domain of systems, wireless/wired networking and security.

%\noindent{\bf 1. Dissertation Work -- Fast and Secure Handoffs Using Neighbor Graphs:}
\section{Dissertation Work -- Fast and Secure Handoffs Using Neighbor Graphs}
As a user moves within a wireless
network (cellular or WLAN), he changes his point-of-attachment (called an association) to the network through a
procedure termed handoff. Handoffs in general, incur certain latencies during which the user is unable to send/receive
application traffic. This causes an adverse affect on stateful network protocols such as TCP and also acts as a major
deterrent for synchronous multimedia applications such as voice and streaming video. My thesis address this problem in
the context of 802.11 WLANs.  

% Much like in the cellular case, handoffs in 802.11 incur certain latencies during which the user is
%unable to send/receive application traffic. This can have adverse affect on protocols such as TCP which might
%incorrectly attribute packet losses during handoffs to network congestion. Also, this might be a big deterrent for
%synchronous multimedia applications such as voice and video which cannot tolerate significant jitter.  My PhD thesis
%addresses the problem posed by handoff latency periods for application traffic. The first step in my research was to
%understand how the handoff process works for 802.11 and to quantify its latencies, which I describe next.

$\bullet$ {\it Analysis of 802.11 Handoff Latencies:} Through careful measurements and experiment designs over
a testbed network  of 40 APs (built off Soekris NET4521/OpenBSD), we performed an in-depth analysis of the handoff
process for 802.11. This analysis resulted in the first and  widely cited paper published at {\it ACM Sigcomm CCR,
2003 \cite{ccr-handoff}}, that explains how handoffs happen in 802.11 networks by presenting the algorithms that various wireless network
interface (NIC) vendors implement, provides insights into where the latencies are incurred, what the latency numbers are
and the factors that affect them. This paper received considerable attention in both the research community and the
industry; it also attracted funding from Samsung Electronics.
Through this analysis, we categorized handoff latencies into two: (i) Scanning: The mobile user scans available 802.11
channels in order to search for APs. This latency was measured between 100 - 400 ms depending on the algorithm
used for scanning. (ii) Authentication: Based on the IEEE 802.1X standard for robust certificate-based authentication,
this phase incurred a high latency of about 800 ms. The combined latency comes to about 1.2 seconds which is far
too high for synchronous multimedia applications and also poses a significant hurdle for everyday TCP-style
applications. 

$\bullet$ {\it Software Artifacts - Open1x.org Effort:} The measurement of the authentication latency was based on my
implementation of the IEEE 802.1X standard.  We made this available as opensource on {\it open1x.org} which was founded
by me in 2002. Today, this code is maintained by a team of enthusiastic  graduate students and is popularly used on Linux
platforms especially Debian. Through this implementation, we also demonstrated the possibility of mounting
Man-in-the-Middle attacks on an 802.1X system. This work gained media publicity ({\it cited on CNN.com, Feb 2002}) and
was also published as a journal paper ({\it Wiley Journal of Wireless Networking, 2004 \cite{security2004}} and downloaded about half-a-million times till date). 
  
$\bullet$ {\it Notion of Neighbor Graphs:} My thesis is based on the central notion of what we call, {\it
neighbor graphs}, defined as a graph structure over the set of APs that comprise a given wireless
network. The edges capture unique mobility properties of the users in the network environment. A directed edge is placed
between two access points {\it if} they act as successive points-of-attachment for users, that is, users handoff between
those two access points. Neighbor graphs, which are built over this unique and simple concept, provided the base for
building a set of system-level solutions to tackle the scanning and the authentication latencies. The initial work was
published at {\it IEEE Infocom 2004 \cite{mishra2004}}.

$\bullet$  {\it Solutions for Fast Authentication and Scanning:} By carefully analyzing the authentication
mechanism used by 802.1X, we built a key distribution scheme (published in {\it Proc. of IEEE Wireless Communications,
2004}) based on neighbor graphs, that works with the 802.1X
standard, the 802.11i wireless security standard and allows for fast re-authentication with the next access point
during handoff. By utilizing standard cryptographic techniques and proofs, we showed that this scheme provides the same
level of strength as the original 802.1X/802.11i mechanism. By implementing this technique over the 40-AP testbed, we 
showed that the latency costs were much less at about 5 ms as compared to the 800 ms cost incurred earlier.  
Using neighbor graphs, we built a fast scanning algorithm (published at {\it ACM Mobisys
2004 \cite{shin2004}}) that reduced the scanning latencies by avoiding wasteful scan operations without compromising on the quality of
the scan results. This was implemented using the Airjack driver for Prism  802.11b wireless cards, and scanning
latencies of under 50 ms were achieved. 

Overall, through my thesis research, I was able to build a full system for fast handoffs that achieved a 50 ms budget
for the handoff latencies (down from 1.2 sec). These solutions did not compromise either on the quality of scan results
or the strength of the 802.1X based authentication methods. Moreover, this system was based on a single and powerful
concept of neighbor graphs which uniquely captures the topology of the wireless network from the angle of user mobility.
This research has given me immense systems expertise and lessons on how to design good practical solutions which I hope
to apply to future research challenges.



\section{Client-driven Channel Management for Wireless LANs} Spectrum is scarce and thus proper utilization of 802.11
channels is critical to the performance of a WLAN. Existing methods for assignment of channels to APs are either static
or are ``AP-centric'', that is they do not capture performance metrics at the clients. My recent work (published 
in {\it ACM MCCR 2005 \cite{graph2005} and Infocom 2006 \cite{infocom2006}}) has shown that such approaches can cause severe
under-utilization of spectrum. In this work, we also designed and implemented a ``client-driven'' approach to channel
management  that uses simple feedback mechanisms from clients to better assign channels to APs. This work highlighted
the broad concept of the necessity of client participation in network management for WLANs.


There are a number of unique aspects of this work compared with previous work.  First, we demonstrated that for best
client performance, the channel assignment problem should be solved {\em in conjunction with} the client-AP association
problem.  Second, our solution approach illustrates some problems associated with a graph-based formulation of the
problem and demonstrates that a {\em set-based}\ formulation better models the different constraints. We expect this
idea to apply in many other wireless scenarios.  Third, we developed centralized solutions that apply to enterprise
WLANs, as well as distributed variants that would co-exist in an environment where multiple WLANs share the same
spectrum. To my knowledge, this was the {\em first paper to define the client-driven approach and demonstrate its
potential}\ for wireless network management in various environments.

In my follow-up work (published at {\it ACM Mobicom 2006 \cite{mobicom2006}}) we tackle the problem of channel
management for  uncoordinated wireless deployments where cooperation between APs cannot be assumed. These entities can
be, for example, neighboring coffee shops, apartments and small businesses who mostly maintain a single-AP WLAN. Such
APs that interfere with each other are collectively faced with the challenge of selecting a good assignment of channels.
We built a fully distributed mechanism to allocate channels which is client-driven and also utilizes
partially-overlapped channels for maximum spectrum use. The solution is built on the concept of channel hopping and is
algorithmically shown to provide a fair division of the spectrum resources among competing APs while maximizing the
overall network and spectrum utilization.

\section{Spectrum Assignment with Partially-overlapped Channels} My next direction of work allows for further
improvement in spectrum utilization beyond channel management by building the theory and techniques to take advantage of
partially overlapped channels apart from non-overlapped ones.  We observe that the 2.4 GHz band defines 3
non-overlapping and 11 partially overlapped channels. In our initial work, ({\it IMC 2005 \cite{pov}}) we demonstrated that the
standard practice of ``avoiding'' partially overlapped channels due to the interference between such channels is
actually causing underutilization of spectrum.  We showed that with proper models of the interference effects in
partially-overlapped channels it might in fact be possible to use such channels to attain a significantly better system
design.

In our followup work ({\it ACM Sigmetrics 2006 \cite{sigm2006}}), we designed an accurate and efficient model of the performance of
partially-overlapped channels for different wireless standards, such as 802.11, 802.16, and so forth. We developed
algorithms that use the model to efficiently assign partially-overlapped channels to wireless interfaces, leading to a
{\em factor of three improvement in achieved throughputs}\ for many hypothesized wireless LAN and mesh network
workloads. Through analysis, simulation, and implementation, this work, thus debunked the previous practice of
restricting channel assignment choices to non-overlapping channels alone. Our proposed model is fairly general; for
example, we demonstrated that using the model, we can take {\em any}\ existing channel assignment algorithm that
currently assigns only non-overlapping channels and derive another channel assignment allowing partially-overlapped
channels. This work has the potential for widespread use in the widely adopted standards for spectrum management for
enterprises, homes, and other wireless hotspots.


\section{Future Directions for Research} With my experience with wireless systems and security, I am keen on 
understanding and solving some of the big challenges we face today as we try to move to a  wireless future
where the user is in control and not the technology. This vision is attainable through a combination of two interesting 
directions of research that I might pursue in near future.

$\bullet$ {\it Fine-grained Spectrum Sharing:} Popular statistics show that over 90\% of the licensed spectrum remains
unused at any point of time.  Such spectrum could alleviate some of the performance issues faced by the WLAN technology
which operates in the crowded unlicensed bands. This leads to the vision that tomorrows devices would be able to make
effective use of even the licensed bands by operating in a cooperative fashion with the incumbent licensed users. There
are many challenges that remain to be addressed both at the wireless and the networking layer before such a system can
be realistically deployed. I hope to be able to understand and build protocol-level methods to engineer the
next-generation of such networks. 

$\bullet$ {\it Convergence through Cognitive Radio technology:} This vision draws support from the congitive radio
technology that is actively being developed as a vehicle to implement one single interface that can allow both voice and
data communication across a wide range of wireless technologies. The classic example is a user being able to have his
cellphone communicate via bluetooth, 802.11 and GSM mechanisms depending on demand. Another related challenge with this
vision of convergence is that of battery-life. Some of the protocols that we use today especially the VoIP stack and
802.11 were not designed with battery consumption in mind. Thus, their usage can limit the battery life for such a
smartphone device and prove to be a major hurdle for commercialization. I hope to address some of these challenges that lie
ahead in realizing this vision.

\iffalse
Another broad focus of my research  has been into managing spectrum/channels for 802.11. I define the term channel management
as the problem of assigning channels to APs and clients to APs so as to maximize the network's throughput and minimize
interference. The underlying observation is simple -- two APs that are within interference range of each other can be
assigned to different non-overlapping channels so that they don't interfere. However, the channels are scarce; there are
3 non-overlapping channels in the 2.4 GHz band. Thus proper assignment of channels to APs is critical to maximising the
utilization of the valuable wireless spectrum. 

Traditionally, channel assignment (in the cellular case) is modeled as a graph coloring problem, where the vertices are
base-stations or access points and an edge indicates interference. One of the key observations that I made, was that
graph coloring based algorithms {\it cannot} do the right assignment of channels as they cannot capture interference
arising due to clients. From here, I built a conflict-set coloring model to explicitly capture interference at the
clients. This model works in an enterprise setting where clients can be assigned to any AP within range. This allowed
for the joint optimization of channel assignment and load balancing of clients among APs. Through NS simulations and
experiments on a 70-node inbuilding testbed at Maryland, I showed how this model supersedes static and graph coloring
based assignment methods. This work was published at IEEE Infocom 2006. 

Another component of my research into channel management has been the demonstration of the ability to take
advantage of partially overlapped channels. There are 11 channels defined in the 2.4 GHz band, out which only
3 are non-overlapping. The default method is to just use only non-overlapping channels. However, as I was performing
experiments, I observed that if the physical separation between two nodes is sufficient enough, they can operate on
partially overlapped channels. From this obervation, I built a model and the underlying theory using which one
can systematically design algorithms that can use partially overlapped channels. This work was published at 
ACM Sigmetrics in 2006.  
 
graph coloring to conflict set coloring - client participation.

partially overlapped channels.

channel hopping.
Interference is a characteristic of the wireless medium that makes it unique and challenging. As clients associate
to access points, they suffer contention/interference from other clients either associated to the same AP  
or to a different access point. By assigning different non-overlapping channels to access points
 My next broad direction
of research over the past few years has been into managing channels in 802.11. 

{\bf Client-driven:}
We propose an efficient client-based approach for
channel management (channel assignment and load balancing)
in 802.11-based WLANs that lead to better usage of the wireless
spectrum. This approach is based on a “conflict set coloring” formulation
that jointly performs load balancing along with channel
assignment. Such a formulation has a number of advantages.
First, it explicitly captures interference effects at clients. Next,
it intrinsically exposes opportunities for better channel re-use.
Finally, algorithms based on this formulation do not depend on
specific physical RF models and hence can be applied efficiently
to a wide-range of in-building as well as outdoor scenarios.
We have performed extensive packet-level simulations and
measurements on a deployed wireless testbed of 70 APs to
validate the performance of our proposed algorithms. We show
that in addition to single network scenarios, the conflict set
coloring formulation is well suited for channel assignment where
multiple wireless networks share and contend for spectrum in
the same physical space. Our results over a wide range of both
simulated topologies and in-building testbed experiments indicate
that our approach improves application level performance at the
clients by upto three times (and atleast 50%) in comparison to
current best-known techniques.

{\bf POV:}
Many wireless channels in different technologies are known to have
partial overlap. However, due to the interference effects among
such partially overlapped channels, their simultaneous use has typically
been avoided. In this paper, we present a first attempt to model
partial overlap between channels in a systematic manner. Through
the model, we illustrate that the use of partially overlapped channels
is not always harmful. In fact, a careful use of some partially
overlapped channels can often lead to significant improvements in
spectrum utilization and application performance. We demonstrate
this through analysis as well as through detailed application-level
and MAC-level measurements. Additionally, we illustrate the benefits
of our developed model by using it to directly enhance the
performance of two previously proposed channel assignment algorithms
— one in the context of wireless LANs and the other in the
context of multi-hop wireless mesh networks. Through detailed
simulations, we show that use of partially overlapped channels in
both these cases can improve end-to-end application throughput by
factors between 1.6 and 2.7 in different scenarios, depending on
wireless node density. We conclude by observing that the notion
of partial overlap can be the right model of flexibility to design efficient
channel access mechanisms in the emerging software radio
platform


{\bf Mobicom:}
Wireless 802.11 hotspots have grown in an uncoordinated fashion
with highly variable deployment densities. Such uncoordinated
deployments, coupled with the difficulty of implementing
coordination protocols, has often led to conflicting configurations
(e.g., in choice of transmission power and channel of operation)
among the corresponding Access Points (APs). Overall, such conflicts
cause both unpredictable network performance and unfairness
among clients of neighboring hotspots. In this paper, we focus on
the fairness problem for uncoordinated deployments. We study
this problem from the channel assignment perspective. Our solution
is based on the notion of channel-hopping, and meets all
the important design considerations for control methods in uncoordinated
deployments — distributed in nature, minimal to zero
coordination among APs belonging to different hotspots, simple
to implement, and interoperable with existing standards. In particular,
we propose a specific algorithm called MAXchop, which
works efficiently when using only non-overlapping wireless channels,
but is particularly effective in exploiting partially-overlapped
channels that have been proposed in recent literature. We also
evaluate how our channel assignment approach complements previously
proposed carrier sensing techniques in providing further
performance improvements. Through extensive simulations on real
hotspot topologies and evaluation of a full implementation of this
technique, we demonstrate the efficacy of these techniques for not
only fairness, but also the aggregate throughput, metrics.
We believe that this is the first work that brings into focus the
fairness properties of channel hopping techniques and we hope that
the insights from this research will be applied to other domains
where a fair division of a system’s resources is an important consideration.


- one of the killer applications for wireless technology is telephony. today 
most people use the cellular wireless technology for voice and 802.11 wireless
for data. this trend is rapidly changing, with voice over ip becoming the mainstream
communications vehicle. 

Wireless Local Area Networks (WLANs) are experiencing un-
precedented growth as the last mile connectivity solution. Mobility is
an important feature of any wireless communication system. Handoffs
are a crucial link level functionality that enable a mobile user to stay
connected to a wireless network by switching the data connection from
one base station or access point to another. Conceptually the handoff
process can be subdivided into two phases: (i) Discovery - wherein the
client searches for APs in vicinity and (ii) Authentication - the client
authenticates to an AP selected from the discovery phase.


       The handoff procedure recommended by the IEEE 802.11 stan-
dard and closely implemented by various wireless vendors is an intrusive
and a brute-force approach. My testbed based study of these algo-
rithms showed that they incur high latencies varying between 400ms to
1.3 seconds depending on the security settings in effect. Such inefficient
handoff mechanisms can have a detrimental impact on applications es-
pecially synchronous multimedia connections such as Voice over IP.
       In my dissertation, I have proposed and evaluated the notion of
locality among APs induced by user mobility patterns. A relation is cre-
ated among APs which captures this locality in a graph theoretic man-
ner called neighbor graphs ­ a distributed structure that autonomously
captures such locality. Based on this, I have designed and evaluated
efficient mechanisms to address the two different phases of this hand-
off process. Through a rigorous testbed based implementation, I have
demonstrated the viability of the concept of mobility induced locality
through good performance improvements. Through extensive simula-
tions I have studied the performance of proposed handoff mechanisms
over various different topologies. This work has shown that a topo-
logical structure which captures the locality relationship among APs is
fundamental to designing mechanisms that make user mobility trans-
parent from the higher layers of the networking stack.

\fi

\bibliographystyle{IEEE}
\bibliography{rs}

\end{document}
