% LaTeX resume using res.sty
\documentstyle[fancyhead]{res} 
% Use \documentstyle[fancyhead,newcent]{res} to get New Century Schoolbook
% Postscript font; the fancyhead option is used to get 2 line header
% Use \documentstyle[fancyhead]{res} to get default (Computer Modern) font
\setlength{\topmargin}{-0.6in}  % Start text higher on the page 
\setlength{\textheight}{9.8in}  % increase textheight to fit more on a page
\setlength{\headrulewidth}{0pt} % suppress line drawn by default with fancyhead
\setlength{\headsep}{0.2in}     % space between header and text
\setlength{\headheight}{12pt}   % make room for header

\lhead{\hspace*{-\sectionwidth}Arunesh Mishra} % force lhead all the way left
\rhead{Page \thepage}  % put page number at right
\cfoot{}  % the footer is empty
\pagestyle{fancy} % set pagestyle for the document
\begin{document} 
\thispagestyle{empty} % this page does not have a header
\name{ARUNESH MISHRA}
\address{\\6204 44th Ave,\\
Riverdale, MD 20737.\\
Email: arunesh@cs.umd.edu\\
(408) 505 - 5086 (cell)\\
(301) 405 - 8162 (off)}


\begin{resume}
\vspace{0.1in}
%\moveleft.5\sectionwidth\centerline{as a research intern for summer 2004.}  

\section{EDUCATION}
\vspace{0.1in} 
{\bf Ph.D., Computer Science}  (December 2005),
    \begin{itemize}
         \item[] Advisor: Dr. William Arbaugh, \\
		 Thesis: {\em Supporting Secure and Transparent Mobility in Wireless LANs} \\
                 University of Maryland.
    \end{itemize}

{\bf M.S., Computer Science},   	
    \begin{itemize}
         \item[] University of Maryland, August 2003.
    \end{itemize}

{\bf B.Tech., Computer Science (President Gold Medalist)},
    \begin{itemize}
          \item[] Indian Institute of Technology, August 2000.
    \end{itemize}

 
\section{RESEARCH INTERESTS} 

\vspace{0.1in}
 Wireless Networks, Network and Systems Security.


\section{EXPERIENCE} 
\vspace{0.1in}
 {\bf June 2001 - current},   	
    \begin{itemize}
         \item[] Research Assistant, (with Dr. William Arbaugh)\\
                 {\em University of Maryland, College Park.}\\
		 Focused on security and mobility issues in high speed Wireless LANs as a part of this dissertation research. 
                 Worked towards building topological information in the form
                 of so called {\em neighbor graphs} and mechanisms that use such structures to make the effects 
                 of user mobility transparent to the higher layers of the networking stack. Such mechanisms address
                 the security and performance issues of handoffs in wireless networks by making them faster and at the same time
                 providing strong security guarantees. Studied the feasibility of neighbor graphs through a full-fledged testbed
		 based implementation and rigorous simulations.

                 Also worked on {\em channel management} issues in wireless LANs. Innovated a novel channel
                 model, called {\em conflict-set} based channel management which address the joint problem of channel assignment and
                 load balancing in the context of wireless infrastructure networks. Demonstrated that this new model
		 and corresponding algorithms performed well in realistic scenarios
                 through testbed based experiments and diverse simulations.

    \end{itemize}

 {\bf June 2004 - Aug 2004},   	
    \begin{itemize}
         \item[] Research Intern, (with Dr. James Kempf)\\
		 {\em DoCoMo Research Labs, San Jose, CA.}

		 Worked on developing a Layer 2.5 protocol that enabled a user's wireless connection to be
                 multiplexed over multiple interfaces potentially connecting to diverse wireless networks thus providing a fully 
                 hybrid wireless networking platform. Designed the protocol to be visible to the applications as a single
                 network interface that managed the real interfaces based on policy or performance parameters. Built a Linux kernel based
                 implementation as a part of this study.
    \end{itemize}

 {\bf May-Aug 2002 },   	
    \begin{itemize}
         \item[] Research Intern, (with Dr. Jonathan Agre)\\
		{\em Fujitsu Labs of America, College Park, MD.}
			
		Worked on the design and implementation of one of the first wireless mesh networks, called the {\em SNOWNET}.
	        The implementation showed the need for dynamic routing, authentication and privacy protocols. Implemented modifications to the IEEE 802.1X Standard
                to support an inductive authentication method. This work was patented and Fujitsu is promoting this project's
                protocols and ideas into the IEEE 802.11s Working Group.
		
    \end{itemize}

 {\bf Aug-Dec 2000 },   	
    \begin{itemize}
         \item[] Teaching Assistant, (with Dr. Michelle Hugue)\\
		{\em University of Maryland, College Park.}\\
		Course : Advanced Data-Structures. 
    \end{itemize}

 {\bf May-Aug 1999 },   	
    \begin{itemize}
         \item[] Research Intern, (with Dr. Natwar Modani)\\
		{\em IBM India Research Labs, New Delhi.}\\
		Worked on designing an internet based auction system that used a {\em push} based mechanism to
                coordinate bidding among competing buyers. Built a full-fledged web based implementation using a database as the
                backend. Also constructed a graph theoretic model and developed bidding algorithms that
                acted as agents on part of users to maximize certain objectives. This project received a best intern award
                and resulted in a patent and a paper.

    \end{itemize}


\section{CONFERENCE PUBLICATIONS}
\vspace{0.5cm}

[1] ``Exploiting Partially Overlapping Channels in Wireless Networks: Turning a Peril into an Advantage'',
Arunesh Mishra, Eric Rozner, Suman Banerjee and William Arbaugh, {\em in the Proceedings of ACM/USENIX Internet Measurement Conference (IMC)}
, Berkeley, CA, October 2005.

[2] ``Eliminating Handoff Latencies in 802.11 WLANs Using Multiple Radios: Applications, Experience, and Evaluation'', 
Vladimir Brik, Arunesh Mishra and Suman Banerjee, {\em in the Proceedings of ACM/USENIX Internet Measurement Conference (IMC)}
, Berkeley, CA, October 2005.


[3] ``Improving the Latency of 802.11 hand-offs using Neighbor Graphs'', Min-ho Shin, Arunesh Mishra, and William A. Arbaugh,
{\em in Proceedings of 2nd International Conference on Mobile Systems, Applications and Services (MobiSys)}, Boston, June 2004.


[4] ``Context Caching Using Neighbor Graphs for Fast Handoffs in a Wireless Network'', Arunesh Mishra, 
Min-ho Shin and William A. Arbaugh, {\em in Proceedings of the 23rd IEEE Conference on
 Computer Communications (INFOCOM)}, Hong Kong, March 2004.

%\vspace{0.1in}


[5] ``Minimizing Broadcast Latency and Redundancy in Ad Hoc Networks'', Rajiv Gandhi, Srinivasan Parthasarathy and
Arunesh Mishra {\em in Proceedings of the Fourth ACM International Symposium on
Mobile Ad Hoc Networking and Computing (MOBIHOC)}, June 2003.

[6] ``Winner Determination in Combinatorial Auctions with restriction on bidding patterns'', Arunesh Mishra and
K. Balaji, {\em in Proceedings of the International Conference on Information Technology }, Bhuwaneshwar, India, 1999.

\section{JOURNAL PUBLICATIONS}
\vspace{0.5cm}

[7] ``Weighted Coloring based Channel Assignment in WLANs'', Arunesh Mishra, Suman Banerjee and William Arbaugh,
in {\em ACM SIGMOBILE Mobile Computing and Communications Review (MC2R)}, July, 2005.


[8] ``Security Issues in IEEE 802.11 Wireless Local Area Networks: A Survey'', Arunesh Mishra, Nick L. Petroni Jr. William Arbaugh and Timothy Fraser, 
in {\em  Wiley Interscience Wireless Communications and Mobile Computing Journal (Wiley Wireless)}, Vol 4 No 8, December, 2004.

[9] ``Proactive Key Distribution Using Neighbor Graphs'', Arunesh Mishra,  Min-ho Shin and William A. Arbaugh,
{\em in IEEE Wireless Communications, February, 2004}. 

[10] ``An Empirical Analysis of the IEEE 802.11 MAC Layer Handoff Process'', Arunesh Mishra,  
Min-ho Shin and William A. Arbaugh, {\em in the ACM SIGCOMM Computer Communication Review (ACM CCR),
Vol 33, No 2, April 2003}. 



\section{OTHERS (Workshops, IEEE Working Group Contributions)}

\vspace{0.5cm}

[11] ``Using Partially Overlapped Channels in Wireless Meshes'', Arunesh Mishra, Suman Banerjee and William Arbaugh,
as an {\em invited paper} at the {\em First IEEE Workshop on Wireless Mesh Networks}, in conjunction with IEEE SECON,
Santa Clara, September 2005.

[12] ``Client-driven Channel Management in Wireless LANs'', Arunesh Mishra, Vladimir Brik, Suman Banerjee, Aravind Srinivasan
and William Arbaugh, as a student poster at  the {\em The Eleventh Annual International Conference on Mobile Computing and Networking (MobiCom)},
Cologne, Germany, September 2005.

[13] ``Inclusion of Optimal-Channel Time into IEEE 802.11k'', Arunesh Mishra,  Min-ho Shin, William Arbaugh and Insun Lee,
at the {\em IEEE 802.11 Working Group Meeting, San Francisco, Document IEEE 802.11-03/541 K, July 2003}.

[14] ``Fast Handoffs Using Fixed Channel Probing'' Arunesh Mishra, Min-ho Shin, William Arbaugh and Insun Lee, at the 
{\em IEEE 802.11 Working Group Meeting, San Francisco, Document IEEE 802.11-03/540 K, July 2003}.

[15] ``Secure-Spaces: Location-based Secure Group Communication for Wireless Networks'', Arunesh Mishra and 
Suman Banerjee, {\em in the ACM MOBICOM Mobile Computing and Communications Review (MC2R)}, Vol. 1, No. 2, October 2002.
Also appears as a student poster in {\em ACM Mobicom}, September 2002.


[16] ``An Initial Security Analysis of the IEEE 802.1X Standard'', Arunesh Mishra and William A. Arbaugh,
{\em Technical Report, University of Maryland, Department of Computer Science CS-TR-4328, 
UMIACS-TR-2002-10}, Feburary 2001, cited on {\em CNN} - Feb 18 2001.


[17] ``Opensource Implementation of IEEE 802.1X Standard'', Arunesh Mishra and William A. Arbaugh,
{\em Work-In-Progress Talk at the Tenth USENIX Security Symposium, August 2001}.


[18] ``The Co-processor as an Independent Auditor'', Arunesh Mishra and Jesus Molina and William A. Arbaugh,
{\em Work-In-Progress Talk at the IEEE Symposium on Security and Privacy, Oakland, CA, 2001}.



\section{HONORS}
\vspace{0.1in}

{\bf President of India Gold Medal - 2000}
    \begin{itemize}
         \item[] Awarded for highest GPA among all 2000-batch undergraduate students at IIT Guwahati.
    \end{itemize}


{\bf Student Rank One Merit Scholarship 1997-2000}
    \begin{itemize}
         \item[] Indian Institute of Technology, Guwahati.
    \end{itemize}

{\bf Best Presentation Award - 1999}
    \begin{itemize}
         \item[] Awarded for best project intern at IBM India Research Labs, New Delhi. 
    \end{itemize}

\section{MISCELLANEOUS}
\vspace{0.1in}

{\bf PATENTS}
    \begin{itemize}
         \item[] ``A Secure Nomadic Wireless Network: SNOWNET'' with Fujitsu Labs of America, MD.
	 \item[] ``Method for Fast Roaming in a Wireless Network'' with Samsung Electronics.
    \end{itemize}

{\bf MEDIA FOCUS}
    \begin{itemize}
         \item[] {\em CNN, PCWorld, ZDNet News, Slashdot},  ``Researchers claim to crack
           wireless security'', Feburary 18, 2002.
         \item[] {\em CNET Asia}, ``Wireless Network security shows cracks'', Feburary 19, 2002.  
         \item[] {\em BusinessWeek Online}, ``This LAN is Whose LAN?'', Feburary 21, 2002. 
	 \item[] {\em IEEE 802.1aa and 802.1X}, Working Group contributions.
    \end{itemize}

{\bf INDEPENDENT REVIEWER} 
    \begin{itemize}
         \item[]  International Conference on Network Protocols (ICNP) 2001.
         \item[]  IEEE Infocom 2003, 2005.
         \item[]  IEEE Globecom 2003.
         \item[]  IEEE Wireless Communications and Networking Conference (WCNC) 2004.
         \item[]  Wireless Communications and Mobile Computing Journal, John Wiley and Sons.
         \item[]  ACM MOBICOM Mobile and Computer Communications Review (MC2R).
	 \item[]  EURASIP Journal of Wireless Communications.
	 \item[]  IEEE Transactions on Mobile Computing.
         \item[]  Wiley Journal on Wireless Networks.
    \end{itemize}

\iffalse
{\bf COMPUTING SKILLS} 
    \begin{itemize}
         \item {\bf Major Skills (5+ years of experience)}
         \begin{itemize}
                \item[] C (7 yrs) and C++ (7 yrs), 
	        \item[] Java (4 yrs),
	        \item[] Shell Programming (5 yrs),
	        \item[] Systems and Network Programming (5 yrs).
         \end{itemize}
         \item {\bf Brief Exposure (project based)}
         \begin{itemize}
                \item[] Visual Basic (1 yr) , Visual C++ (1 yr), HTML(1 yr), COBOL, Access Basic, PL-SQL, Lisp, Prolog.
         \end{itemize}
         \item {\bf Platforms}
         \begin{itemize}
                \item[] UNIX(5yrs), HP-UX(4yrs), Linux(6yrs), Windows (5 yrs) (9x, NT, 2000, XP), IBM-AIX, Solaris(2 yrs).
         \end{itemize}

    \end{itemize}
\fi
 
\section{REFERENCES}
   \begin{itemize}
        \item[] Dr. William Arbaugh, University of Maryland, (waa@cs.umd.edu).
	\item [] Dr. Bobby Bhattacharjee, University of Maryland (bobby@cs.umd.edu).
	\item [] Dr. Jonathan Agre, Fujitsu Labs of America, (jagre@fla.fujitsu.com).
	\item[] Dr. Jonathan Katz, University of Maryland, (jkatz@cs.umd.edu).
   \end{itemize}
\end{resume}
\end{document}






























