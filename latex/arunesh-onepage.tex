% LaTeX resume using res.sty
\documentstyle[fancyhead]{res} 
% Use \documentstyle[fancyhead,newcent]{res} to get New Century Schoolbook
% Postscript font; the fancyhead option is used to get 2 line header
% Use \documentstyle[fancyhead]{res} to get default (Computer Modern) font
\setlength{\topmargin}{-0.6in}  % Start text higher on the page 
\setlength{\textheight}{9.8in}  % increase textheight to fit more on a page
\setlength{\headrulewidth}{0pt} % suppress line drawn by default with fancyhead
\setlength{\headsep}{0.2in}     % space between header and text
\setlength{\headheight}{12pt}   % make room for header

\lhead{\hspace*{-\sectionwidth}Arunesh Mishra} % force lhead all the way left
\rhead{Page \thepage}  % put page number at right
\cfoot{}  % the footer is empty
\pagestyle{fancy} % set pagestyle for the document
\begin{document} 
\thispagestyle{empty} % this page does not have a header
\name{ARUNESH MISHRA}
\address{\\4140 A.V. Willams Bldg,\\
College Park, MD 20742.\\
Email: arunesh@cs.umd.edu\\
(408) 505 - 5086 (cell)}


\begin{resume}
\vspace{0.1in}
%\moveleft.5\sectionwidth\centerline{as a research intern for summer 2004.}  

\section{EDUCATION}
\vspace{0.1in} 
{\bf Ph.D., Computer Science}  (ongoing), M.S.(August 2003),
    \begin{itemize}
         \item[] Advisor: Dr. William Arbaugh, \\
		 Thesis: {\em Supporting Secure and Transparent Mobility in Wireless LANs}, \\
                 University of Maryland.
    \end{itemize}

\section{EXPERIENCE} 
\vspace{0.1in}
    \begin{itemize}
         \item[] Research Assitant, {\em University of Maryland}, June 2000 - current.
         \item[] Research Intern, {\em DoCoMo Research Labs, CA}, Summer 2004.
         \item[] Research Intern, {\em Fujitsu Labs of America, MD}, Summer 2002.
         \item[] Research Intern, {\em IBM India Research Labs, India}, Summer 1999.
     \end{itemize}

\section{SELECTED PUBLICATIONS}
\vspace{0.5cm}

[1] ``Exploiting Partially Overlapping Channels in Wireless Networks: Turning a Peril into an Advantage'',
Arunesh Mishra, Eric Rozner, Suman Banerjee and William Arbaugh, {\em in the Proceedings of ACM/USENIX Internet Measurement Conference (IMC)}
, Berkeley, CA, October, 2005.

[2] ``Eliminating Handoff Latencies in 802.11 WLANs Using Multiple Radios: Applications, Experience, and Evaluation'', 
Vladimir Brik, Arunesh Mishra and Suman Banerjee, {\em in the Proceedings of ACM/USENIX Internet Measurement Conference (IMC)}
, Berkeley, CA, October, 2005.

[3] ``Weighted Coloring based Channel Assignment in WLANs'', Arunesh Mishra, Suman Banerjee and William Arbaugh,
in {\em ACM SIGMOBILE Mobile Computing and Communications Review (MC2R)}, July, 2005.



[4] ``Improving the Latency of 802.11 hand-offs using Neighbor Graphs'', Min-ho Shin, Arunesh Mishra, and William A. Arbaugh,
{\em in Proceedings of 2nd International Conference on Mobile Systems, Applications and Services (MobiSys)}, Boston, June, 2004.


[5] ``Context Caching Using Neighbor Graphs for Fast Handoffs in a Wireless Network'', Arunesh Mishra, 
Min-ho Shin and William A. Arbaugh, {\em in Proceedings of the 23rd IEEE Conference on
 Computer Communications (INFOCOM)}, Hong Kong, March, 2004.

[6] ``Security Issues in IEEE 802.11 Wireless Local Area Networks: A Survey'', Arunesh Mishra, Nick L. Petroni Jr. William Arbaugh and Timothy Fraser, 
in {\em  Wiley Interscience Wireless Communications and Mobile Computing Journal (Wiley Wireless)}, Vol 4, No 8, December, 2004.

[7] ``Proactive Key Distribution Using Neighbor Graphs'', Arunesh Mishra,  Min-ho Shin and William A. Arbaugh,
{\em in IEEE Wireless Communications}, February, 2004. 

[8] ``An Empirical Analysis of the IEEE 802.11 MAC Layer Handoff Process'', Arunesh Mishra,  
Min-ho Shin and William A. Arbaugh, {\em in the ACM SIGCOMM Computer Communication Review (ACM CCR)},
Vol 33, No 2, April, 2003. 


\section{REFERENCES}
   \begin{itemize}
        \item[] Dr. William Arbaugh, University of Maryland, (waa@cs.umd.edu).
	\item [] Dr. Bobby Bhattacharjee, University of Maryland, (bobby@cs.umd.edu).
   \end{itemize}
\end{resume}
\end{document}






























