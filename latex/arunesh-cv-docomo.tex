% LaTeX resume using res.sty
\documentstyle[fancyhead]{res} 
% Use \documentstyle[fancyhead,newcent]{res} to get New Century Schoolbook
% Postscript font; the fancyhead option is used to get 2 line header
% Use \documentstyle[fancyhead]{res} to get default (Computer Modern) font
\setlength{\topmargin}{-0.6in}  % Start text higher on the page 
\setlength{\textheight}{9.8in}  % increase textheight to fit more on a page
\setlength{\headrulewidth}{0pt} % suppress line drawn by default with fancyhead
\setlength{\headsep}{0.2in}     % space between header and text
\setlength{\headheight}{12pt}   % make room for header

\lhead{\hspace*{-\sectionwidth}Arunesh Mishra} % force lhead all the way left
\rhead{Page \thepage}  % put page number at right
\cfoot{}  % the footer is empty
\pagestyle{fancy} % set pagestyle for the document
\begin{document} 
\thispagestyle{empty} % this page does not have a header
\name{ARUNESH MISHRA}
\address{7811 Mandan Road, \# 301\\
Greenbelt, MD 20770\\
Email: arunesh@cs.umd.edu\\
(301) 477 - 2305 (res)\\
(301) 405 - 8162 (off)}


\begin{resume}
\vspace{0.1in}
%\moveleft.5\sectionwidth\centerline{as a research intern for summer 2004.}  

\section{EDUCATION}
\vspace{0.1in} 
{\bf Ph.D., Computer Science}  (ongoing),
    \begin{itemize}
         \item[] Advisor: Dr. William A. Arbaugh, \\
		 Area: Fast and Secure Roaming in Wireless Networks,\\
                 University of Maryland.
    \end{itemize}

{\bf M.S., Computer Science},   	
    \begin{itemize}
         \item[] University of Maryland, August 2003.
    \end{itemize}

{\bf B.Tech., Computer Science},
    \begin{itemize}
          \item[] Indian Institute of Technology, August 2000.
    \end{itemize}

 
\section{RESEARCH INTERESTS} 

\vspace{0.1in}
 Wireless Networks, Network and Systems Security, Embedded Systems, Peer-to-peer and distributed systems.


\section{EXPERIENCE} 
\vspace{0.1in}
 {\bf June 2001 - current},   	
    \begin{itemize}
         \item[] Research Assistant (with Dr. William A. Arbaugh) 20 hrs/week \\
                 University of Maryland.
    \end{itemize}

 {\bf May-Aug 2002 },   	
    \begin{itemize}
         \item[] Research Intern, (with Dr. Jonathan Agre) 40 hrs/week \\
	         Fujitsu Labs of America, MD.
    \end{itemize}

 {\bf Aug-Dec 2000 },   	
    \begin{itemize}
         \item[] Teaching Assistant (with Dr. Michelle Hugue) 20 hrs/week \\
                 University of Maryland.
    \end{itemize}

 {\bf May-Aug 1999 },   	
    \begin{itemize}
         \item[] Research Intern, (with Dr. Natwar Modani) 40 hrs/week \\
                 IBM India Research Labs, New Delhi.
    \end{itemize}


\section{PUBLICATIONS}
\vspace{0.1in}

``Context Caching Using Neighbor Graphs for Fast Handoffs in a Wireless Network'', {\em with} 
Min-ho Shin and William A. Arbaugh, {\em in Proceedings of the 23rd IEEE Conference on
 Computer Communications (INFOCOM) 2004}. Also as a {Technical Report, UMD, CS-TR-4477, UMIACS-TR-2003-46}.


``Improving the Latency of 802.11 hand-offs using Neighbor Graphs'' {\em with} Min-ho Shin and William A. Arbaugh, as a
{\em Technical Report, University of Maryland, Department of Computer Science CS-TR-4550, December 2003}.


``An Empirical Analysis of the IEEE 802.11 MAC Layer Handoff Process'' {\em with}
Min-ho Shin and William A. Arbaugh, {\em in the ACM SIGCOMM Computer Communication Review (ACM CCR),
Volume 33, Issue 2, April 2003}. Also as a {Technical Report, UMD, CS-TR-4395, UMIACS-TR-2002-75}.


``Proactive Key Distribution Using Neighbor Graphs'' {\em with} Min-ho Shin and William A. Arbaugh,
{\em in IEEE Wireless Communications Magazine, Feburary, 2004}. Also as a {\em Technical Report, UMD,  CS-TR-4538, UMIACS-TR-2002-106}.


``Inclusion of Optimal-Channel Time into IEEE 802.11k'' {\em with}  Min-ho Shin, William A. Arbaugh and Insun Lee,
at the {\em IEEE 802.11 Working Group Meeting, San Francisco, Document IEEE 802.11-03/541 K, July 2003}.


``Fast Handoffs Using Fixed Channel Probing'' {\em with} Min-ho Shin, William A. Arbaugh and Insun Lee, at the 
{\em IEEE 802.11 Working Group Meeting, San Francisco, Document IEEE 802.11-03/540 K, July 2003}.


``Minimizing Broadcast Latency and Redundancy in Ad Hoc Networks'' {\em with} 
Rajiv Gandhi and Srinivasan Parthasarathy, {\em in Proceedings of the Fourth ACM International Symposium on
Mobile Ad Hoc Networking and Computing (MOBIHOC), June 2003}.

``Secure-Spaces: Location-based Secure Group Communication for Wireless Networks'' {\em with}
Suman Banerjee, {\em in the ACM MOBICOM Mobile Computing and Communications Review, Vol. 1, No. 2, October 2002}.
Also appears as a student poster in {\em ACM Mobicom, September 2002}.


``An Initial Security Analysis of the IEEE 802.1X Standard'', {\em with} William A. Arbaugh,
{\em Technical Report, University of Maryland, Department of Computer Science CS-TR-4328, 
UMIACS-TR-2002-10, Feburary 2001}.


``Opensource Implementation of IEEE 802.1X Standard'', {\em with} William A. Arbaugh,
{\em Work-In-Progress Talk at the Tenth USENIX Security Symposium, August 2001}.

``The Co-processor as an Independent Auditor'', {\em with} Jesus Molina and William A. Arbaugh,
{\em Work-In-Progress Talk at the IEEE Symposium on Security and Privacy, Oakland, CA, 2001}.

``Winner Determination in Combinatorial Auctions with restriction on bidding patterns'' {\em with}
K. Balaji, {\em in Proceedings of the International Conference on Information Technology 1999, India}.

``Design and Implementation of a Geographic Information System'', as the {\em B.Tech. Technical Report,
 Indian Institute of Technology, 2000}.


\section{ACHIEVEMENTS / HONORS}
\vspace{0.1in}

{\bf President of India Gold Medal - 2000}
    \begin{itemize}
         \item[] For Academic Excellence during undergraduate degree program, Indian Institute of Technology, Guwahati.
    \end{itemize}


{\bf Student Rank One Merit Scholarship 1997-2000}
    \begin{itemize}
         \item[] Indian Institute of Technology, Guwahati.
    \end{itemize}

{\bf Best Presentation Award - 1999}
    \begin{itemize}
         \item[] IBM India Research Labs, New Delhi. 
    \end{itemize}

\section{MISCELLANEOUS}
\vspace{0.1in}

{\bf PATENTS}
    \begin{itemize}
         \item[] ``A Secure Nomadic Wireless Network: SNOWNET'' with Fujitsu Labs of America, MD.
	 \item[] ``Method for Fast Roaming in a Wireless Network'' with Samsung Electronics.
    \end{itemize}

{\bf MEDIA FOCUS}
    \begin{itemize}
         \item[] {\em CNN, PCWorld, ZDNet News, Slashdot},  ``Researchers claim to crack
           wireless security'', Feburary 18, 2002.
         \item[] {\em CNET Asia}, ``Wireless Network security shows cracks'', Feburary 19, 2002.  
         \item[] {\em BusinessWeek Online}, ``This LAN is Whose LAN?'', Feburary 21, 2002. 
    \end{itemize}

{\bf REVIEWER} 
    \begin{itemize}
         \item[]  International Conference on Network Proctocols (ICNP) 2001.
         \item[]  IEEE Infocom 2003.
         \item[]  IEEE Globecom 2003.
         \item[]  IEEE Wireless Communications and Networking Conference (WCNC) 2004.
         \item[]  Wireless Communications and Mobile Computing Journal, John Wiley and Sons.
         \item[]  ACM MOBICOM Mobile and Computer Communications Review (MC2R).
    \end{itemize}

{\bf COMPUTING SKILLS} 
    \begin{itemize}
         \item {\bf Major Skills (5+ years of experience)}
         \begin{itemize}
                \item[] C (7 yrs) and C++ (7 yrs), 
	        \item[] Java (4 yrs),
	        \item[] Shell Programming (5 yrs),
	        \item[] Systems and Network Programming (5 yrs).
         \end{itemize}
         \item {\bf Brief Exposure (project based)}
         \begin{itemize}
                \item[] Visual Basic (1 yr) , Visual C++ (1 yr), HTML(1 yr), COBOL, Access Basic, PL-SQL, Lisp, Prolog.
         \end{itemize}
         \item {\bf Platforms}
         \begin{itemize}
                \item[] UNIX(5yrs), HP-UX(4yrs), Linux(6yrs), Windows (5 yrs) (9x, NT, 2000, XP), IBM-AIX, Solaris(2 yrs).
         \end{itemize}
    \end{itemize}
 
\end{resume}
\end{document}






























