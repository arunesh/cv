% LaTeX resume using res.sty
\documentstyle[fancyhead]{res} 

% Use \documentstyle[fancyhead,newcent]{res} to get New Century Schoolbook
% Postscript font; the fancyhead option is used to get 2 line header
% Use \documentstyle[fancyhead]{res} to get default (Computer Modern) font
\setlength{\topmargin}{-0.6in}  % Start text higher on the page 
\setlength{\textheight}{9.3in}  % increase textheight to fit more on a page
\addtolength{\textwidth}{-0.1in}  % increase textheight to fit more on a page
\setlength{\headrulewidth}{0pt} % suppress line drawn by default with fancyhead
\setlength{\headsep}{0.2in}     % space between header and text
\setlength{\headheight}{12pt}   % make room for header

\newdimen\mylinewidth 
\setlength{\mylinewidth}{\resumewidth}
\addtolength{\mylinewidth}{-0.4in}

\def\bull{\vrule height 0.8ex width .7ex depth -.1ex }
%\newcommand{\lineunder}{\vspace*{-8pt} \\ \hspace*{-18pt} \hrulefill \\}
\newcommand{\lineunder}{\vspace*{-8pt} \\ \hspace*{-18pt} \rule{\mylinewidth}{0.5pt} \\}
\newcommand{\justline}{\hspace*{-18pt} \hrulefill \\\vspace{4pt}}
\newcommand{\header}[1]{{\hspace*{-15pt}\vspace*{2pt} \large{\textsc{#1}}} \vspace*{-2pt} \lineunder}
\newcommand{\contact}[3]{
\vspace*{-8pt}
\begin{center}
{\LARGE \scshape {#1}}
\lineunder
#2\\
#3
\end{center}
\vspace*{-8pt}
}

\lhead{\hspace*{-\sectionwidth}} % force lhead all the way left
\rhead{Page \thepage}  % put page number at right
\cfoot{}  % the footer is empty
\pagestyle{fancy} % set pagestyle for the document
\begin{document} 
\contact{A r u n e s h \hspace{3mm}    M i s h r a}
{  San Francisco Bay Area.}
{aruneshm@gmail.com}
\thispagestyle{empty} % this page does not have a header

%\name{ARUNESH MISHRA}
%\address{\\6204 44th Ave,\\
%Riverdale, MD 20737.\\
%Email: arunesh@cs.umd.edu\\
%(408) 505 - 5086 (cell)\\
%(301) 405 - 8162 (off)}


\begin{resume}
%\moveleft.5\sectionwidth\centerline{as a research intern for summer 2004.}  
 
\header{Interests}
Peer-peer networking and distributed system protocols for Blockchain Infrastructure.
Deep Learning for mobile applications.
Wireless and Wired Networking Systems, Systems Security, Design of Scalable and Robust
Systems, Location Inferencing from Wireless Networks, Location Security and Privacy.

\header{Work Experience}
{\bf Feb 2017 - date}
    \begin{itemize}
        \item [] {\it Advisor and Mentor}, Zhongguancun Innovation Center, Santa Clara. \\
            Advising startups on ML/Deep Learning technologies. Worked with 4 different startups on network design,
            strategy and pitch events. Also gave mutiple talks on the topic at events.
        \item [] {\it Advisor}, Latitude Blockchain Startup. \\ 
            Designing the Latitude protocol, blockchain and core
            whitepaper. Latitude is a blockchain that fundamentally shifts how data is shared in the Transportation
            Industry (Mapping, Telematics, Insurance).
        \item [] {\it Research/Exploratory projects.} \\
            Designed and implemented an AI app (30K daily active users, 4.4 rating) that fixes painful problems on Android phones. 
            Working on a proposal for {\em Safely} -- an deep learning-based app for user safety vigilance. Working on
            {\em Picolo} (https://picolo.network) a decentralized scalable database for blockchain apps.
    \end{itemize}
{\bf Jan 2013 - Feb 2017}
    \begin{itemize}
         \item[] {\it Tech Lead, Software Engineer}, Google, Mountain View, CA.\\

             Launched a network stack for peer-peer gaming on iOS and Android via Google Play Games SDK. Designed and
             implemented the network stack that maintains performance under changing network conditions and NAT
             topologies. Worked with the CEO of VectorUnit for RipTide demo at Google I/O to showcase the platform.
             Launched an API called Google Nearby that can transmit small amount of data securely over near-ultrasound
             for mobile phones. This can be used to authenticate people in same physical room and securely measure physical distance
             between phones to under 1 inch accuracy.
             Formed a team under Marissa Mayer to build an Android app that can automatically determine if two people
             are in the same physical room.
    \end{itemize}

{\bf Jan 2008 - Feb 2013}
    \begin{itemize}
         \item[] {\it Research Engineer}, Google, Mountain View, CA.\\

             Designed a purely crowdsourced-data based scalable system that computes user location using Wifi and Cellular base
             stations. System self-learns from Android users with a GPS fix to push location to indoor-users using
             statistical and randomized signal propagation algorithms. 
	 Conducted research in the area of software defined radios and
	 their application to urban environments for sensing mobile users and
	 traffic. Built cellular remap detection and localization technology, which
	 powers Google's Mobile Maps software (1B+ downloads, most popular
	 app on mobile).
	 
	 System is now used on Android, WinMo and Blackberry phones. Also works on browsers using HTML 5 Geolocation API, such
	 as on Mozilla's Firefox and Chrome browser, also launched with Google's Toolbar for Microsoft's Internet Explorer.
            Algorithms serves over 3 Billion location computations per day on Android and powers Google's mobile Ads
            business unit. The success of the Wifi project helped grow the team to 70 Engineers (from 3).

    \end{itemize}

{\bf Dec 2005 - Oct 2007},
    \begin{itemize}
         \item[] {\it Postdoc Research Scientist}, University of Wisconsin, Madison\\
	        $\bullet$ Designed an energy efficient VoIP over WiFi stack that
		enhances battery lifetime for wireless phones by a factor of 3.
		The new protocol stack uses innovative networking and systems
		concepts designed during the course of this research and have
		wider applicability. Filed three patents capturing the
		innovations in the field.\\
		$\bullet$ Worked on the problem of channel management (continued work from Maryland), published
		results at Sigmetrics and Mobicom. Initiated work on new ideas for WLAN architectures, collision detection and software-defined radios.\\
                $\bullet$ Managed a team of seven undergrad and graduate students working on a multitude of different research
projects  in the areas of WiFi, software-defined radios, city-wide mesh
networks and wireless emulation systems. Conducted world-class research and
published at top tier conferences and journals. Supervised research is being
pursued by current graduate students towards their PhD dissertations. Participated in activities
leading to the procurement of funding from government and private companies. Continue to collaborate and guide the
students till date. This research has won 3 best research awards at the ACM competitions over successive years.
%. Projects include channel management and power control in 802.11, research-level applications of the GNU radio, design 
%and implementation of inbuilding infrastructure and mesh testbeds, design of a monitoring system for a city-wide mesh
%network in Madison and realization of a scalable wireless network emulator system.
    \end{itemize}

 {\bf June 2004 - Aug 2004},   	
    \begin{itemize}
         \item[] {\it Research Intern},  NTT DoCoMo Research Labs, San Jose, CA.\\
		 Worked on developing a Layer 2.5 protocol that enabled a user's wireless connection to be
                 multiplexed over multiple wireless interfaces for bandwidth and QoS gains.
		 
		% Designed the system to be visible to 
                %the applications as a single
                % network interface that managed the underlying wireless interfaces based on policy or performance parameters. Built a Linux kernel based
                % implementation as a part of this study.

   \end{itemize}

 {\bf May-Aug 2002},   	
    \begin{itemize}
         \item[] {\it Research Intern}, Fujitsu Labs of America, College Park, MD.\\
		Worked on the design and implementation of one of the first wireless mesh network projects, called the {\em SNOWNET}.
	        The implementation showed the need for dynamic routing, authentication and privacy protocols. Implemented modifications to the IEEE 802.1X Standard
                to support an inductive authentication method for securing the mesh nodes. This work was patented 7,634,230 and Fujitsu is promoting this project's
                protocols and ideas into the IEEE 802.11s Working Group.
		
    \end{itemize}

 {\bf May-Aug 1999},   	
    \begin{itemize}
         \item[] {\it Research Intern},  IBM India Research Labs, New Delhi.\\
		Worked on designing an internet based auction system that used a {\em push} based mechanism to
                coordinate bidding among competing buyers. Built a full-fledged web based implementation using a database as the
                backend. Also constructed a graph theoretic model and developed bidding algorithms that
                acted as agents on part of users to optimize certain
		objectives. Received a best intern award and published results
		in a reputed conference in India.
    \end{itemize}

\header{Education}
{\bf Ph.D., Computer Science},
    \begin{itemize}
         \item[] Advisor: Dr. William Arbaugh, \\
		 Thesis: {\em Supporting Secure and Transparent Mobility in Wireless LANs} \\
                 University of Maryland, December 2005.
    \end{itemize}

{\bf M.S., Computer Science},   	
    \begin{itemize}
         \item[] University of Maryland, August 2003.
    \end{itemize}

{\bf B.Tech., Computer Science (President Gold Medalist)},
    \begin{itemize}
          \item[] Indian Institute of Technology, August 2000.
    \end{itemize}


\header{Patents}\vspace{-0.7cm}

{\bf Issued:}
    \begin{itemize}
       \item [] [9,380,424] Crowdsourced signal propagation model.
       \item [] [9,377,523] Determining wireless access point locations using clustered data points.
       \item [] [9,355,387] System and method for event management and information sharing.
       \item [] [9,098,589] Geographic annotation of electronic resources.
       \item [] [8,989,101] Systems, methods, and computer-readable media for identifying virtual access points of
           wireless networks.
       \item [] [8,977,265] Method for performing handoff in wireless network.
       \item [] [8,954,094] Mobile device functions based on transportation mode transitions.
       \item [] [8,914,483] System and method for event management and information sharing.
       \item [] [8,868,762] Efficient proximity detection.
       \item [] [8,838,103] Method for performing handoff in wireless network.
       \item [] [8,837,360] Determining geographic location of network hosts.
       \item [] [8,830,909] Methods and systems to determine user relationships, events and spaces using wireless
           fingerprints.
       \item [] [8,825,078] Probabilistic estimation of location based on wireless signal strength.
       \item [] [8,719,426] Efficient proximity detection.
       \item [] [8,706,142] Probabilistic estimation of location based on wireless signal strength and platform
           profiles.
       \item [] [8,688,041] Methods and apparatus for secure, portable, wireless and multi-hop data networking.
       \item [] [8,498,592] Method and apparatus for improving energy efficiency of mobile devices through energy
           profiling based rate adaptation.
       \item [] [8,478,280] Minimum coverage area of wireless base station determination.
       \item [] [8,229,442] Derivation of minimum coverage of cellular sectors with cellular-site spatial density and
           application specific data.
       \item [] [8,218,493] System and method for interference mitigation in wireless networks.
       \item [] [8,150,367] System and method of determining a location based on location of detected signals.
       \item [] [8,085,719] System and method for achieving wireless communications with enhanced usage of spectrum
           through efficient use of overlapping channels.
       \item [] [8,064,921] Method and system for client-driven channel management in wireless communication networks.
       \item [] [8,005,114] Method and apparatus to vary the transmission bit rate within individual wireless packets
           through multi-rate packetization.
       \item [] [7,929,948] Method for fast roaming in a wireless network.
       \item [] [7,881,667] Methods and apparatus for secure, portable, wireless and multi-hop data networking.
       \item [] [7,421,268] Method for fast roaming in a wireless network.
       \item [] [7,400,604] Probing method for fast handoff in WLAN.
       \item [] [7,263,357] Method for fast roaming in a wireless network.
     \end{itemize}

{\bf 22 Patents Pending.}

\newcommand{\mybullet}{\noindent$\circ$~~}
\header{Journal Publications}\vspace{-0.7cm}

\mybullet ``Minimizing broadcast latency and redundancy in ad hoc networks'', Rajiv Gandhi, Arunesh Mishra, Srinivasan Parthasarathy, in
{\em ACM Trans. on Networking}, 2008.

\mybullet ``Deconstructing wireless errors: collision or 'bad' channel?'', Shravan  Rayanchu, Dheeraj Agrawal, Sharad Saha, Arunesh Mishra, Suman Banerjee,
at {\em ACM Mobicom Mobile Computing and Communications Review}, 2008.

\mybullet ``Wireless Security and Interworking'', Minho Shin, Justin Ma, Arunesh Mishra and William Arbaugh in IEEE Cryptography and Security, 2005.

\mybullet ``Weighted coloring based channel assignment in WLANs'', Arunesh Mishra, Suman Banerjee and William Arbaugh,
in {\em ACM Mobicom Mobile Computing and Communications Review}, July, 2005.

\mybullet ``Security issues in IEEE 802.11 WLANs: A survey'', Arunesh Mishra, Nick L. Petroni Jr. William Arbaugh and Timothy Fraser, 
in {\em  Wiley Interscience Wireless Communications and Mobile Computing Journal }, Vol 4 No 8, December, 2004.

\mybullet ``Proactive key distribution using Neighbor Graphs'', Arunesh Mishra,  Min-ho Shin and William A. Arbaugh,
{\em in IEEE Wireless Communications, February,} 2004.

\mybullet ``An empirical analysis of the IEEE 802.11 MAC layer handoff process'', Arunesh Mishra,  
Min-ho Shin and William A. Arbaugh, {\em in the ACM SIGCOMM Computer Communication Review, April} 2003. 

\header{Conference Publications}\vspace{-0.7cm}

\mybullet ``CENTAUR: Realizing the Full Potential of Centralized WLANs through a Hybrid Data Path.'', Vivek Shrivastava, Nabeel Ahmed, Shravan Rayanchu, Suman Banerjee, Srinivasan Keshav, Konstantina Papagiannaki, Arunesh Mishra,
in {\it ACM Mobicom}, Beijing, September, 2009. {\bf Best Paper Award.}

\mybullet ``Supporting Continuous Mobility through Multi-rate Wireless Packetization.'', Arunesh Mishra, Shravan Rayanchu, Dheeraj Agrawal, Suman Banerjee, 
in {\it ACM HotMobile}, Napa Valley, CA, February 2008.

\mybullet ``Diagnosing wireless packet losses in 802.11: Separating collision from weak signal'', Sharavan Rayanchu, Arunesh Mishra, Dheeraj Agrawal, Sharad Saha,
Suman Banerjee, in {\it IEEE Infocom}, Phoenix, AZ, 2008.

\mybullet ``Understanding the limitations of transmit power control for indoor WLANs'',Vivek Shrivastava, Dheeraj Agrawal, Arunesh Mishra, Suman Banerjee, Tamer Nadeem, in {\it ACM Internet Measurement Conference}, San Diego, CA, 2007.

\mybullet ``Interference mitigation in WLANs with speculative scheduling'', Nabeel Ahmed, Vivek Shrivastava, Arunesh Mishra,
   Suman Banerjee, Srinivasan Keshav, Dina Papagiannaki, in  {\it ACM Mobicom}, Montreal, Canada, 2007.

\mybullet ``Deconstructing Wireless Errors : Collision or 'Bad' Channel? '', Shravan Rayanchu, Arunesh Mishra, Dheeraj Agrawal, Sharad Saha, Suman Banerjee,
in {\it ACM Mobicom}, Montreal, Canada, 2007. Won the first prize in the {ACM Student Research Competition}.

\mybullet ``On the (In)feasibility of Fine Grained Power Control'', Vivek Shrivastava, Dheeraj Agrawal, Arunesh Mishra, Suman Banerjee, and Tamer Nadeem,
at the {\it ACM Mobicom,  September 2006}, won the first prize in an {ACM Student Research Competition}.

\mybullet ``Distributed channel management in uncoordinated wireless environments'', Arunesh Mishra, Vivek Shrivastava,
Dheeraj Agrawal, Suman Banerjee, in {\em ACM Mobicom}, 2006.

\mybullet ``Partially overlapped channels not considered harmful'', Arunesh Mishra, Vivek Shrivastava, Suman Banerjee, William
Arbaugh, in {\em ACM Sigmetrics}, 2006.

\mybullet ``Client-driven channel management in wireless LANs'', Arunesh Mishra, Vladimir Brik, Suman Banerjee, Aravind
Srinivasan, William Arbaugh, in {\em IEEE Infocom}, 2006.

\mybullet ``Exploiting partially overlapping channels in wireless networks: Turning a peril into an advantage'',
Arunesh Mishra, Eric Rozner, Suman Banerjee, William Arbaugh, in  {\em  ACM/USENIX Internet Measurement Conference}, 2005.

\mybullet ``Eliminating handoff latencies in 802.11 WLANs using multiple radios: Applications, experience, and evaluation'', 
Vladimir Brik, Arunesh Mishra, Suman Banerjee, in {\em ACM/USENIX Internet Measurement Conference}, 2005.

\mybullet ``Improving the latency of 802.11 hand-offs using Neighbor Graphs'',
Min-ho Shin, Arunesh Mishra, William Arbaugh, in {\em ACM Mobisys},  2004.

\mybullet ``Context caching using Neighbor Graphs for fast handoffs in a wireless network'', Arunesh Mishra, 
Min-ho Shin, William Arbaugh, in {\em  IEEE Infocom}, 2004.

\mybullet ``Minimizing broadcast latency and redundancy in ad-hoc networks'', Rajiv Gandhi, Srinivasan Parthasarathy, 
Arunesh Mishra, in {\em ACM Mobihoc}, 2003.

\mybullet ``Winner determination in combinatorial auctions with restriction on bidding patterns'', Arunesh Mishra, 
K. Balaji, in {\em ICIT }, Bhuwaneshwar, India, 1999.

\header{Workshops and Contributions to IEEE Standards }\vspace{-0.7cm}

\mybullet ``Towards secure localization using wireless `Congruity' '', Arunesh Mishra, Shravan Rayanchu, Ashutosh Shukla, Suman Banerjee
at the {\it IEEE HotMobile Workshop}, Tucson, AZ, 2007.

\mybullet ``TRAC: An Architecture for Real-Time Dissemination of Vehicular Traffic Information'',
Shravan Rayanchu, Sulabh Agarwal, Arunesh Mishra, Suman Banerjee, Samrat Ganguly,  in {\it ACM Mobicom}, Los Angeles CA, 2006.

\mybullet ``Using partially overlapped channels in wireless meshes'', Arunesh Mishra, Suman Banerjee and William Arbaugh,
as an {\em invited paper} at the {\em First IEEE Workshop on Wireless Mesh Networks}, in conjunction with IEEE SECON,
Santa Clara, September 2005.

\mybullet ``Client-driven channel management in wireless LANs'', Arunesh Mishra, Vladimir Brik, Suman Banerjee, Aravind Srinivasan
and William Arbaugh, as a student poster at  the {\em ACM Mobicom}, Cologne, Germany, September 2005.

\mybullet ``Inclusion of optimal-channel time into IEEE 802.11k'', Arunesh Mishra,  Min-ho Shin, William Arbaugh and Insun Lee,
at the {\em IEEE 802.11 Working Group Meeting, San Francisco, Document IEEE 802.11-03/541 K, July 2003}.

\mybullet ``Fast handoffs using fixed channel probing'' Arunesh Mishra, Min-ho Shin, William Arbaugh and Insun Lee, at the 
{\em IEEE 802.11 Working Group Meeting, San Francisco, Document IEEE 802.11-03/540 K, July 2003}.

\mybullet ``Secure-spaces: Location-based secure group communication for wireless networks'', Arunesh Mishra and 
Suman Banerjee, {\em in the ACM MOBICOM Mobile Computing and Communications Review (MC2R)}, Vol. 1, No. 2, October 2002.
Also appears as a student poster in {\em ACM Mobicom}, September 2002.


\mybullet ``An initial security analysis of the IEEE 802.1X Standard'', Arunesh Mishra and William A. Arbaugh,
{\em Technical Report, University of Maryland, Department of Computer Science CS-TR-4328, 
UMIACS-TR-2002-10}, Feburary 2001, cited on {\em CNN} - Feb 18 2001. Over {\em 300,000} downloads.


\mybullet ``Opensource implementation of the IEEE 802.1X Standard'', Arunesh Mishra and William A. Arbaugh,
{\em Work-In-Progress Talk at the Tenth USENIX Security Symposium, August 2001}.


\mybullet ``The co-processor as an independent auditor'', Arunesh Mishra and Jesus Molina and William A. Arbaugh,
{\em WiP at the IEEE Symposium on Security and Privacy, Oakland, CA, 2001}.


\header{Research Conducted at Wisconsin\hfill \textsc{(2005 -- 2007)}}
{\bf Multi-experiment functionality for the ORBIT testbed}
    \begin{itemize}
	\item [] Orbit is an NSF collaborative project focused on the creation of a large-scale wireless network testbed
to allow researchers worldwide to execute  multiple wireless experiments in a
confined and controlled  physical environment.  Funded by NSF, the goal of our
project was to facilitate the execution of multiple wireless experiments
through temporal partitioning of the resources between experiments. Supervised two
undergraduate students through the design and implementation of this system
which was incorporated into ORBIT.
    \end{itemize}
\vspace{-0.2cm}
{\bf Analysis of the performance of a city-wide mesh network. }
    \begin{itemize}
	\item [] This project was in collaboration with a local company, {\it MadCity
 Broadband}, which has deployed a city-wide mesh network to provide Internet access through WiFi
for the city of Madison. Designed and built a distributed monitoring
infrastructure to measure and analyze the performance of their mesh network.
Worked with two graduate students and the concerned employees of the company to
build this infrastructure and install it within the city.
    \end{itemize}
\vspace{-0.2cm}

{\bf Stiglan: An Architecture for centralized scheduling in 802.11 wireless LANs}
    \begin{itemize}
	\item [] Research focusses on the idea of centrally scheduling packets
	in an enterprise 802.11 network. The centralized scheduler maintains a
	network-wide view of interference from other networks and users and
	performs intelligent scheduling of transmissions at the access points
	to minimize collisions and improve throughput. Initial version appears
	as an extended abstract in ACM Mobicom 2007. Final system, renamed to
	``Centaur'' won the {\em Best Paper Award} at ACM Mobicom 2009.
    \end{itemize}
\vspace{-0.2cm}

{\bf Practical transmit power control for 802.11}

\begin{itemize}
\item [] Power control allows for reduced interference and improved battery life.
However, in dynamic wireless conditions such as a WiFi Hotspot, transmit power
needs to be constantly adapted to sustain these gains. Project studies dynamic
adaptation of transmit power parameter through extensive experimentation in
diverse environments and details guildelines for fine-grained transmit power
control algorithms. Initial version appeared as a poster in ACM Mobicom 2006;
full version of this work was published at the ACM Internet Measurement
Conference, 2007.
\end{itemize} \vspace{-0.2cm}

{\bf TRAC -- An architecture for real-time traffic information dissemination}
    \begin{itemize}
	\item [] This research resulted in the design of TRAC -- an architecture to support dissemination of location-sensitive real-time traffic information
specifically targeted towards highly mobile users (vehicles moving at high speeds). Project involved design and
simulation based study of the algorithms and system-level issues involved in implementing such a system. Key innovations
include the notion of virtual publisher and subscriber aggregation mechanisms.
Work was published at a workshop in IEEE Infocom 2007.
    \end{itemize}

\vspace{-0.2cm}

{\bf Secure localization using wireless congruity}
    \begin{itemize}
	\item [] This research resulted in the creation of a concept called
	wireless congruity -- the property that two wireless nodes in vicinity of each other will
experience similar wireless channel characteristics such as packet losses, collisions and contention. Project involves the design
and implementation of a network-wide system based on congruity to provide secure location authentication. Initial
version of this work appears as a paper in ACM HotMobile 2007.
    \end{itemize}
\vspace{-0.2cm}

{\bf Woice: An energy efficient protocol stack for VoIP over 802.11}
    \begin{itemize}
	\item [] IEEE 802.11 uses a strategy based on multiple retransmissions  to evaluate the cause of packet losses
-- collision versus signal degradation. This evaluation has direct impact on
bandwidth and energy resources of the communicating wireless devices. Designed
a novel {\it voice-aware} link layer
adapation mechanism that uses feedback from access points to `fine-tune' the
link layer parameters while consuming significantly
lower energy.  Filed for three patents convering the innovations.
\end{itemize}
\vspace{-0.2cm}

{\bf Spark: An architecture for spectrum slicing in software defined radios}
 \begin{itemize}
 \item [] This research
involves the design and realization of a practical spectrum management and sharing architecture that takes advantage of
recent developments in software radio technology to allow for better spectrum utilization through fine-grained sharing.
Innovated key architectural constructs to solve the problems of spectrum fragmentation (to improve utilization) and
enforcing spectrum leases (security issues) through the design of a system, called {\it Spark}. Initial version appears
as a paper in ACM HotMobile 2007.
 \end{itemize}

\header{PhD Dissertation and Related Research \hfill (2003 -- 2005)}
{\bf Channel management in wireless LANs}
    \begin{itemize}
	\item [] This research addressed the joint challenge of assigning
	radio-frequency channels to 802.11 access points (APs) and allocating
	users among APs, called channel management. Developed a conflict-based set coloring model which allows clients to `drive' the channel assignment
decisions made at the APs.  Built practical centralized and distributed solutions separately tailored for both enterprise and hotspot
environments. Through carefully studied experiments, showed the ability to use partially overlapped channels in both 2.4
and 5 GHz band. This directly improved spectrum utilization by 30\%.  Work was published at premier conferences such as
IMC, Infocom, Mobicom and Sigmetrics.
    \end{itemize}
\vspace{-0.2cm}

{\bf Neighbor graphs for fast handoffs in 802.11 wireless LANs}
    \begin{itemize}
    \item []
  	Built practical and competitive algorithmic solutions for fast and secure 802.11 handoffs in wireless LANs.
       Proposed solutions were studied through a 40-node inbuilding testbed based implementation. They achieved 
       a target handoff latency of under 50 ms from about 400 ms (no security) or 1.2 seconds (802.1X security).
	They have been incorporated into the IEEE 802.11f (mobility) and 802.11i (security) standards apart from being
       patented by Samsung. Work was published at peer-reviewed conferences including Infocom and Mobisys.
    \end{itemize}
\vspace{-0.2cm}
{\bf Open1x.org effort}
    \begin{itemize}
	\item [] Engineered the first and widely used opensource implementation of the IEEE 802.1X standard available at
{\it www.open1x.org} which provides for robust wireless authentication. This has been incorporated into a number of linux
and embedded OS distributions including Debian. Also discovered and rectified critical security flaws in this standard. 
Work received considerable media focus, including CNN and was published as a journal paper. Suggested fixes were
incorporated in the IEEE 802.1X-2004 version.
    \end{itemize}


\header{Recognition / Awards}
{\bf Best paper award, Mobicom, 2009}
    \begin{itemize}
         \item[] Awarded for the paper titled, ``Centaur: Realizing the Full Potential of Centralized WLANs through a Hybrid Data Path.".
    \end{itemize}

{\bf Google's Executive Management Group's Innovation Award, 2008}
    \begin{itemize}
         \item[] Awarded by the Company's highest level Management for innovative work in
	 Location-based services for Mobile phones. Award was given to less than 30 Employees from a total of 10,000.
    \end{itemize}

{\bf President of India Gold Medal - 2000}
    \begin{itemize}
         \item[] Awarded for highest GPA among all 2000-batch undergraduate students at IIT Guwahati.
    \end{itemize}


{\bf Student Rank One Merit Scholarship 1997-2000}
    \begin{itemize}
         \item[] Indian Institute of Technology, Guwahati.
    \end{itemize}

{\bf Best Intern Award - 1999}
    \begin{itemize}
         \item[] Awarded for best project intern at IBM India Research Labs, New Delhi. 
    \end{itemize}

{\bf Business Plan Competition - Runners up, 2002}
    \begin{itemize}
         \item[] Won a runners-up award for a business plan competition at the
	 Robert H. Smith School of Business at the University of Maryland. The
	 business was to build a portable MP3 player with Wifi capability.
    \end{itemize}

{\bf My work directly cited in the Media}
    \begin{itemize}
         \item[$\bullet$] {\em Google Deducing Wireless Data}, on Slashdot and other Tech websites, Jan 29, 2010.
         \item[$\bullet$] {\em CNN, PCWorld, ZDNet News, Slashdot},  ``Researchers claim to crack
           wireless security'', Feburary 18, 2002.
         \item[$\bullet$] {\em CNET Asia}, ``Wireless network security shows cracks'', Feburary 19, 2002.  
         \item[$\bullet$] {\em BusinessWeek Online}, ``This LAN is Whose LAN?'', Feburary 21, 2002. 
    \end{itemize}

{\bf Recognition in the Professional Community}
    \begin{itemize}
	\item  [$\bullet$] {\em Invited as a reviewer:} Invited as an external
	reviewer for all major conferences and journals including IEEE Infocom,
	ACM ICNP, IEEE Globecom, IEEE WCNC, ACM Mobicom, ACM MC2R, IEEE Transactions on Mobile
	Computing, IEEE Transactions in Networking and the Wiley Journal of Wireless Networks.
        \item [$\bullet$] {\em Technical program committee member} for
	conferences: The 14th International Confernce on High Performance
	Computing (HiPC), 2007; the 14th The International Conference on
	Advanced Computing and Communication (ADCOM), 2007; and the Fifth Annual
	IEEE Communications Society Conference on Sensor, Mesh and Ad Hoc
	Communications and Networks (SECON), 2008, ACM Mobicom's Mobiheld, 2009.
     \end{itemize}
\iffalse
         \item[]  International Conference on Network Protocols (ICNP) 2001.
         \item[]  IEEE Infocom 2003, 2005.
         \item[]  IEEE Globecom 2003.
         \item[]  IEEE Wireless Communications and Networking Conference (WCNC) 2004.
         \item[]  Wireless Communications and Mobile Computing Journal, John Wiley and Sons.
         \item[]  ACM MOBICOM Mobile and Computer Communications Review (MC2R).
	 \item[]  EURASIP Journal of Wireless Communications.
	 \item[]  IEEE Transactions on Mobile Computing.
         \item[]  Wiley Journal on Wireless Networks.
    \end{itemize}
\fi

%\header{References}
%   \begin{itemize}
%	\item[] Available upon request.
%   \end{itemize}

%   \begin{itemize}
%	\item[] $\bullet$ Dr. Suman Banerjee, University of Wisconsin, (suman@cs.wisc.edu).
%        \item[] $\bullet$ Dr. William Arbaugh, University of Maryland, (waa@cs.umd.edu).
%	\item [] $\bullet$ Dr. Bobby Bhattacharjee, University of Maryland (bobby@cs.umd.edu).
%	\item [] $\bullet$ Dr. Jonathan Agre, Fujitsu Labs of America, (jagre@fla.fujitsu.com).
%   \end{itemize}
\end{resume}
\end{document}






























